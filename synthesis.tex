\chapter{Synthesis}
\label{synthesis}
\newpage

\section{Past}

\noindent

\dropcap{T}{his} thesis started with two very basic biological 
questions. The first question is to ask which species are closest related to
one another. This can be answered satifactorily 
nowadays using DNA sequences: the taxa of
which the DNA sequences is most similar, are closest related
\footnote{
  Even this simple rule-of-thumb has a hidden assumption:
  the mutation rate should be low enough to preserve genetic information,
  instead of scrambling all nucleotides.
}.
Already for decades, biologists have created multiple phylogenetic tools 
that allow the user to construct such an 
undated (that is, without an indicator of time) 
phylogeny from a DNA alignment.
A novel such tool that I developed is \verb;babette; (chapter 2), 
which allows to do a Bayesian phylogenetic inference from R,
resulting in a distribution of undated (but also possibly dated) phylogenies.


The second basic biological question, is to 
ask \emph{when} these speciation events took place,
so that we can get an idea of which species (co-)existed when.
Constructing \emph{a} dated phylogeny from a DNA alignment 
using \emph{a} tool is easy.
It is harder to get a reasonably accurate dated phylogeny,
because one needs to make a defendable pick from the many combinations of 
models. To make a defendable pick, however, we can sometimes use
a rational procedure. With \verb;mcbette; (a component of \verb;pirouette;,
chapter 3), one can do a model
comparison, which concludes which inference model is best picked
on an alignment. 

Even would we know the model underlying a phylogeny, we
still make an error in constructing our dated phylogeny: due
to chance, the resulting DNA alignment may have 
too less or too much mutations in certain places.

This brings us to the bigger question underlying this thesis:
How well can we construct a phylogeny from an alignment?
With theoretical studies, in which we have complete knowledge,
we can measure the error in the phylogenies constructed.

In nature, there is a bigger problem: 
we do not know, which phylogenetic model it follows.
Instead, we incorporate different parts
of the phylogenetic process in different models, 
each of which may not apply or be relevant in different circumstances.

For example, we can either assume that each species has the same
mutation rate, or that each species has
its own unique mutation rate.
It seems to be the more cautious approach to
assume each species has a unique mutation rate: 
if all these unique mutations rates are close, we fall back
to the other model.
And, indeed, there is evidence from nature that we should 
assume a different mutation rate per species [A General Comparison of Relaxed Molecular Clock Models. Thomas Lepage,*David Bryant,Herve ́ Philippe,àand Nicolas Lartillot].
An open question is, however, how these mutation rates are distributed.
However, for phylogenies that go back little in time (hence
have few mutations) and have little rate variation,
we are better off assuming a mutation rate that is the same in each
species [Rate variation and estimation of divergence times using strict and relaxed clocks, Richard P Brown & Ziheng Yang],
as our estimates will have higher certainty.

The part of a phylogenetic model I focussed on in this thesis,
is the speciation model. The speciation model encorporates
a phylogeny's branching pattern. 
For example, we can either assume that through time,
each species has the same probability to speciate, 
or that each species has
its own unique speciation rate.
It seems to be the more cautious approach to
assume each species has a unique speciation rate: 
if all these unique speciation rates are close, we fall back
to the other model.


Also here, we do not know what is the
true speciation model in nature.

Due to this, there are many speciation models, each
incorporating a facet of the speciation process. When we have to
pick a defensible speciation model, we are usually pragmatic and
pick a model that is incorporated within our phygenetic
tools. There are cases, however, in which we could worry that these
standard models may be overly simplistic, because they overlook a 
facet of speciation that may be very convincing in some cases.

In this thesis, I describe a tool I created together with Giovanni Laudanno, 
to measure the error we make in inferring a phylogeny, due to
picking the wrong speciation model. 
With this novel tool, called \verb;pirouette; (chapter 3),
the field of phylogenetics now has a method, 
with which we demonstrate if and when the standard speciation models
are justifiably good. 

I define 'standard speciation model' as the speciation models 
present in the phylogenetic tool BEAST2, which are the Yule and
Birth-Death model. The Yule model assumes no extinction
and a constant speciation rate, where the Birth-Death model
assumes a constant speciation and extinction rate. As
it is 

There are two speciation models that are (yet) non-standard.
The first speciation model, 
unlike the standard speciation models, 
allows speciation to co-occur.
When there is a scenario in nature, in which a process triggers speciation
in multiple species, this multiple-birth death (MBD) model would be the better fit. 
In chapter 4, Giovanni Laudanno and me 
use \verb;pirouette; to 
measure the error we make, 
when we use a standard speciation model on a process we know (read: simulated) 
to follow the MBD process.
From that we found that ...

The second non-standard speciation model,
unlike the standard speciation models,
assumes that speciation takes time,
which is encorporated in the protracted-birth death (PBD) model
In chapter 5, I 
use \verb;pirouette; to 
measure the error we make, when we use a standard speciation
model on a process we know to follow the PBD process.
From that we found that ...

\section{Open questions and future work}

Now we can measure the errors we make in our phylogentic
inference, when using a standard phylogentic model on a
true process following a non-standard model. 
\verb;pirouette; can serve as a litmus test 
to measure the relevance of a novel tree prior. 

Of course, there are already ways to estimate the relevance of a
novel speciation model. A straighforward one, is to add the speciation
model to the set of standard models, after which its inference is
compared with the standard models. A drawback of this approach, is
that it is harder to develop: not only need the novel speciation model
be able to simulate its phylogenies, also a likelihood equation is needed
to allow it to be used by the phylogenetic tools. Developing such a
likelihood equation may be harder than using \verb;pirouette;.

The step forward of \verb;pirouette; is that it allows 
to quantify the error we make in our phylogentic inference, by
expressing it as (usually) two distributions: one with the baseline errors,
one with the added error caused by using an incorrect phylogenetic model.
What is missing in \verb;pirouette;, is a clear-cut interpretation of these 
error distributions, that is, a clear yes/no answer to the question: 'Is
using the standard phylogenetic models good enough?'. Only when the two
distributions overlap, can we confidently claim a yes. 

There have been defensible, yet arbitrary, choices made in the pipeline
of \verb;pirouette;. For example, a twin alignment has as much
mutations acculated from the root sequence, as the true alignment. 
This design choice should assure the Bayesian inference in the next step
to work on an equal amount of genetic information. Unknown is if this
indeed improves the judgements made by \verb;pirouette;. These choices
should one day be parameters of using \verb;pirouette; as well.

We apply the \verb;pirouette; methodology in chapter 4 and 5.
We show that the error in a Bayesian inference on
phylogenies increases when the effects of co-occuring 
or protracted speciation are stronger. 
This qualitative conclusion is obvious, 
but the extent of these errors has never been quantified before.
An obvious follow-up question is, 
what are examples in nature where we expect a high 
abundance of co-occurring or protracted speciation, 
and thus, what is the
error empiricist make?

Disregarding the results of chapters 3 and 4, there are
at least two reasons to add the MBD and PBD tree models to the standard
phylogentic models anyway. The first reason is methodological: perhaps
the \verb;pirouette; setup gave rise to an inrepresentative conclusion.
Using the actual tree prior in inference and comparing its performance
to the standard ones, is still the most straightforward way to determine,
ironically, if the novel tree prior adds value as a standard tree prior.
The second reason to add a novel tree prior the standard models, is
because of the model parameters that are estimated jointly with the
phylogenies. The clearest example is the PBD model, which allows one
to estimate biologically relevant parameters such as the duration of speciation.

This thesis gives the field of phylogenetics tools and examples of
how to quantify the effect of a tree prior on Bayesian inference.
It will help research find the border between when speciation models are
simple enough, yet not too simple. As all articles within this thesis,
this thesis itself and all its source code is free (as in freedom),
there is little in the way of this research contributing to the field. 

\references{conclusion}

