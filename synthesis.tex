\chapter{Synthesis}
\label{synthesis}
\newpage

\section{Past}

\noindent

## Goal 

The goal of the research projects
described in this thesis 
was to measure the error we make in today's phylogenetic inference.
In the end, I must conclude that the foundation to do so, is firmly 
put in place, reasonably well investigated and paves the way for
more biological example.

## Discussion of results

### pirouette and razzo only lightly explored


## Apply to PBD model: raket

## Mechanistic model: pbdmms

## Mainland extinction: daisieme

## Altmetrics

Some altmetrics:

 * 60,000 GitHub commits, 1.1k repos, 421 stars, 242 followers
 * 133 YouTube videos on babette (9), R (8), C++ (14), 68 subscribers, 11k views
 * Supervised 2 MSc students (Jorik Boer, Kees Wesselink)
 * Supervised 6 BSc students (Joris Damhuis, 
   Damian Over, Dave Nijhuis, Elles Jetten, Femke Thon, Jolien Gay)
 * Supervised 7 interns from secondary schools (Jorn Prenger, Mart Prenger, 
   Anne Hinrichs, Joshua van Waardenberg, Marijn Meerveld, Jeroen Niemendal, 
   Rijk van Putten)
 * Organised 172 social events
 * Since Jan 2017, presented 20 times at TECE
 * Publish 5 packages on CRAN (beautier, tracerer, beastier, mauricer, babette)
 * Passed rOpenSci peer-review for 4 R packages (beautier, tracerer, beastier, babette)
 * Taught +220 evenings about programming

\dropcap{T}{his} thesis answers a facet of the question 


'What is the error we make in inferring a phylogeny

describes a tool I created together with Giovanni Laudanno, 
to measure the error we make in inferring a phylogeny, due to
picking the wrong speciation model. 
With this novel tool, called \verb;pirouette; (chapter 3),
the field of phylogenetics now has a method, 
with which we demonstrate if and when the standard speciation models
are justifiably good. 

I define 'standard speciation model' as the speciation models 
present in the phylogenetic tool BEAST2, which are the Yule and
Birth-Death model. The Yule model assumes no extinction
and a constant speciation rate, where the Birth-Death model
assumes a constant speciation and extinction rate. As
it is 

There are two speciation models that are (yet) non-standard.
The first speciation model, 
unlike the standard speciation models, 
allows speciation to co-occur.
When there is a scenario in nature, in which a process triggers speciation
in multiple species, this multiple-birth death (MBD) model would be the better fit. 
In chapter 4, Giovanni Laudanno and me 
use \verb;pirouette; to 
measure the error we make, 
when we use a standard speciation model on a process we know (read: simulated) 
to follow the MBD process.
From that we found that ...

The second non-standard speciation model,
unlike the standard speciation models,
assumes that speciation takes time,
which is encorporated in the protracted-birth death (PBD) model
In chapter 5, I 
use \verb;pirouette; to 
measure the error we make, when we use a standard speciation
model on a process we know to follow the PBD process.
From that we found that ...

\section{Open questions and future work}

Now we can measure the errors we make in our phylogentic
inference, when using a standard phylogentic model on a
true process following a non-standard model. 
\verb;pirouette; can serve as a litmus test 
to measure the relevance of a novel tree prior. 

Of course, there are already ways to estimate the relevance of a
novel speciation model. A straighforward one, is to add the speciation
model to the set of standard models, after which its inference is
compared with the standard models. A drawback of this approach, is
that it is harder to develop: not only need the novel speciation model
be able to simulate its phylogenies, also a likelihood equation is needed
to allow it to be used by the phylogenetic tools. Developing such a
likelihood equation may be harder than using \verb;pirouette;.

The step forward of \verb;pirouette; is that it allows 
to quantify the error we make in our phylogentic inference, by
expressing it as (usually) two distributions: one with the baseline errors,
one with the added error caused by using an incorrect phylogenetic model.
What is missing in \verb;pirouette;, is a clear-cut interpretation of these 
error distributions, that is, a clear yes/no answer to the question: 'Is
using the standard phylogenetic models good enough?'. Only when the two
distributions overlap, can we confidently claim a yes. 

There have been defensible, yet arbitrary, choices made in the pipeline
of \verb;pirouette;. For example, a twin alignment has as much
mutations acculated from the root sequence, as the true alignment. 
This design choice should assure the Bayesian inference in the next step
to work on an equal amount of genetic information. Unknown is if this
indeed improves the judgements made by \verb;pirouette;. These choices
should one day be parameters of using \verb;pirouette; as well.

We apply the \verb;pirouette; methodology in chapter 4 and 5.
We show that the error in a Bayesian inference on
phylogenies increases when the effects of co-occuring 
or protracted speciation are stronger. 
This qualitative conclusion is obvious, 
but the extent of these errors has never been quantified before.
An obvious follow-up question is, 
what are examples in nature where we expect a high 
abundance of co-occurring or protracted speciation, 
and thus, what is the
error empiricist make?

Disregarding the results of chapters 3 and 4, there are
at least two reasons to add the MBD and PBD tree models to the standard
phylogentic models anyway. The first reason is methodological: perhaps
the \verb;pirouette; setup gave rise to an inrepresentative conclusion.
Using the actual tree prior in inference and comparing its performance
to the standard ones, is still the most straightforward way to determine,
ironically, if the novel tree prior adds value as a standard tree prior.
The second reason to add a novel tree prior the standard models, is
because of the model parameters that are estimated jointly with the
phylogenies. The clearest example is the PBD model, which allows one
to estimate biologically relevant parameters such as the duration of speciation.

This thesis gives the field of phylogenetics tools and examples of
how to quantify the effect of a tree prior on Bayesian inference.
It will help research find the border between when speciation models are
simple enough, yet not too simple. As all articles within this thesis,
this thesis itself and all its source code is free (as in freedom),
there is little in the way of this research contributing to the field. 

\references{synthesis}

